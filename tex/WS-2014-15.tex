% Kopfzeile beim Kapitelanfang:
\fancypagestyle{plain}{
%Kopfzeile links bzw. innen
\fancyhead[L]{\Large Wintersemester 2014/15}
%Kopfzeile rechts bzw. außen
\fancyhead[R]{}}
%Kopfzeile links bzw. innen
\fancyhead[L]{\Large Wintersemester 2014/15}
%Kopfzeile rechts bzw. außen
\fancyhead[R]{}
% **************************************************
\phantomsection
\addcontentsline{toc}{chapter}{Inhaltsverzeichnis für das Wintersemester 2014/15}
\chapter*{Inhaltsverzeichnis für das Wintersemester 2014/15}
Dieses Inhaltsverzeichnis erhebt keinen Anspruch auf Vollständigkeit!\\
Bitte besuchen Sie trotzdem die Vorlesung und Übungen, machen Sie die Heimübungen\\
und arbeiten Sie mit dem Buch \href{http://www.amazon.de/Analysis-Differential--Integralrechnung-Ver\%C3\%A4nderlichen-Mathematik/dp/3658003162/ref=sr\_1\_1?ie=UTF8\&qid=1414666359\&sr=8-1\&keywords=analysis+1+forster}{\textit{"`Analysis 1"' von Forster}}.
\section*{Einführung, Vorlesungsüberblick:}
\subsection*{2014-10-13 (Mo)}
\items{
\item Zahlenmystik: \href{http://de.wikipedia.org/wiki/Zahlensymbolik}{\textit{\underline{Wikipedia}}}
\item Assoziativ-, Kommutativ-, Distibutivgesetze: \ref{3.2}
\item Mengenlehre: \ref{Mengen}
\item Fibonacci: \ref{Fibonacci}
}

\subsection*{2014-10-16 (Do)}
\items{
\item Aussagen: \ref{Aussagen}
\item Mengenlehre: \ref{Mengen}
\item Abbildungen zwischen Mengen, Bild, Urbild: \ref{4.1}
\item Potenzmengen: \ref{1.7}
\item Vollständige Induktion: \ref{2.2}
}

\subsection*{2014-10-20 (Mo)}
\items{
\item Wurzelziehen: \ref{3.16}
\item Newton-Verfahren: \ref{Newton-Verfahren}
\item Rationale Zahlen: \ref{Rationale Zahlen}
\item \textit{"`Der kritische Mensch"', \href{http://de.wikipedia.org/wiki/Attac}{attac}, Gemeinnützigkeit, Freihandel, "`Einheitsbrei der Parteien"'}
\item Dezimalbruchdarstellung reeller Zahlen: \ref{8.2}
\item Folgen, Grenzwerte (nächste Woche): \ref{P5}
\item Mengenlehre (\href{http://de.wikipedia.org/wiki/Zermelo-Fraenkel-Mengenlehre}{Zermelo-Fraenkel-Axiome}, leere Menge, Paarbildungsaxiom): \ref{Mengen} und \href{http://de.wikipedia.org/wiki/Zermelo-Fraenkel-Mengenlehre}{\textit{\underline{Wikipedia}}}
\item Kartesisches Produkt: \ref{1.6}
}

\subsection*{2014-10-23 (Do)}
\items{
\item Exponentialfunktion: \ref{8.8}, \ref{10.5}, \href{https://de.wikipedia.org/wiki/Exponentialfunktion}{\textit{\underline{Wikipedia}}}
\item Fakultät: \ref{2.5}
\item differenzierbare Funktionen, Ableitungen: \ref{11.1}
\item Stammfunktion: \ref{14.9}
\item Funktionalgleichung: \ref{8.10}
\item Konvergenz, Grenzwert: \ref{5.2}
\item natürlicher Logarithmus: \ref{10.5}
\item Trigonometrie: \ref{P12}
\item Komplexe Zahlen: \ref{P6}
\item Gruppen: \href{https://de.wikipedia.org/wiki/Gruppe\_(Mathematik)}{\textit{\underline{Wikipedia}}}
\items{
\item Gruppenaxiome: Siehe Körperaxiome (\ref{3.2}), G1 = K3, G2 = K4
}
\item Komposition von Abbildungen: \ref{4.3}, \ref{4.4}
\item Beispiele für Abbildungen (Identität, konstante Abbildung, Bijektion): \ref{BeispieleAbbildungen}
\item Axiomatie für $\R$: \ref{AxiomeReelleZahlen}
\items{
\item Körperaxiome: \ref{Koerperaxiome}
\item Anordnumsaxiome: \ref{Anordnungsaxiome}
\item Vollständigkeitsaxiom: \ref{3.9}
}
}

\subsection*{2014-10-27 (Mo)}
\items{
\item Abgebrochene geometrische Reihe: \ref{2.4}, \ref{lim_xhochn}
\item Binomialkoeffizienten \& bin. Formel: \ref{2.7}, \ref{2.9}, \ref{binFormel}
\items{
\item Rekursionsformel: \ref{2.8}
\item Binomischer Satz und Beweis: \ref{2.10}
}
\item \textit{Dieter Nuhr, freies Land, "`political correctnes"', "`Neger"', "`man darf ja nix mehr sagen"'}
\item Gruppen: \href{https://de.wikipedia.org/wiki/Gruppe\_(Mathematik)}{\textit{\underline{Wikipedia}}}
\items{
\item Gruppenaxiome: Siehe Körperaxiome (\ref{3.2}), G1 = K3, G2 = K4
\item Kommutative / abelsche Gruppe: $x \circ y = y \circ x$ ($x,y \in G$)
}
\item Körper: \ref{3.2}
\items{
\item Endlicher Körper: \href{https://de.wikipedia.org/wiki/Endlicher\_K\%C3\%B6rper}{\textit{\underline{Wikipedia}}}, \ref{BspEndlKoerper}
}
}

\subsection*{2014-10-30 (Do)}
\items{
\item Körper: \ref{3.2}
\items{
\item Angeordnete Körper: \ref{3.5}
\items{
\item Endliche Körper $\F_p$ sind nicht anordbar.
}
\item Anordnungsaxiome: \ref{Anordnungsaxiome}
\item Jeder ang. Körper enthält $\Q$: \href{https://de.wikipedia.org/wiki/Geordneter\_K\%C3\%B6rper\#Strukturaussagen}{\textit{\underline{Wikipedia}}}
\item Archimedisches Axiom: \ref{3.14}
\items{
\item: Folgerungen: \ref{3.15}
\item: $\Q$ ist archimedisch angeordnet.
}
}
\item Betrag: \ref{3.7}
\items{
\item Dreiecksungleichung: \ref{3.8}
}
\item Gauß-Klammer: \ref{BeispieleAbbildungen}
\item Demnächst: Vollständigkeitsaxiom (\ref{3.9}), Cauchy-Folge (\ref{5.13})
\item Bernoulli-Ungleichung: \ref{3.6}
}

\subsection*{2014-11-03 (Mo)}
\items{
\item Mengenlehre (Zermalo-Fraenkel): \href{http://de.wikipedia.org/wiki/Zermelo-Fraenkel-Mengenlehre}{\textit{\underline{Wikipedia}}}
\item Abbildungen $\longleftrightarrow$ Graphen: \ref{4.1}
\item Folgen: \ref{P5}
\items{
\item Konstante, (un-)beschränkte, alternierende (Häufungspunkte) Folge: \ref{5.1}, \ref{5.3}
\item Teilfolgen: \ref{Teilfolgen}
\item Konvergenz, Grenzwert: \ref{5.3}
\items{
\item Beispiele für kon-/divergente Folgen: \ref{5.4}
}
}
\item \textit{Programm}:
\items{
\item Eindeutigkeit des Grenzwerts
\item Rechenregeln für Folgen: \ref{5.7}
\item Beschränktheit: \ref{5.5}
}
}

\subsection*{2014-11-06 (Do)}
\items{
\item Grenzwerte: \ref{5.3}
\item Rechenregeln für Folgen: \ref{5.7}
\item Beschränktheit: \ref{5.5}
\items{
\item Konvergenz $\Ra$ Beschränktheit: \ref{5.6}
}
\item Cauchy-Folgen: \ref{5.13}
\items{
\item Konvergenz $\Ra$ Cauchy-Folge: \ref{5.14}
}
\item Teilfolgen: \ref{5.11}
\item \textit{Demnächst:}
\items{
\item Bolzano-Weierstraß: \ref{5.12}
\item Vollständigkeit
\item Reelle Zahlen: \ref{P3}
}
}

\subsection*{2014-11-10 (Mo)}
\items{
\item Konvergenz $\Lra$ Cauchy-Folge: \ref{5.14} + \ref{3.9}
\item Bolzano-Weierstraß: \ref{5.12}
\item Monotonie: \ref{5.9}
\items{
\item Monotoniekriterium: \ref{5.10}
}
\item Reihen: \ref{P7}
\items{
\item Unendliche Reihen, Konvergenz: \ref{7.1}
\item Cauchy-Kriterium für Konvergenz von Reihen: \ref{7.3}
\item Geometrische Reihe, alternierende Reihe, Harmonische Reihe: \ref{7.2}, \ref{7.7}
\item Majorantenkriterium: \ref{7.11}
\item $\sum_n^\infty a_n$ konvergent $\Ra$ $a_n$ Nullfolge
\items{
\item Gegenrichtung gilt nicht, Bsp: harm. Reihe
}
\item Quotientenkriterium: \ref{7.13}
}
}

\subsection*{2014-11-13 (Do)}
\items{
\item Reihen: \ref{P7}
\item Leibniz-Kriterium: \ref{7.8}
\item Exponentialfunktion: \ref{8.8}, \ref{8.9}
\items{
\item Funktionalgleichung: \ref{8.10}
\item $\exp(x)>0\ \forall x \in \R$ mit $\exp(-x)=\frac{1}{\exp(x)}$: \ref{8.11} (2)
\item $exp: \R \to \R_{>0}, x \mto \exp(x)$ bijektiv
}
\item Cauchy-Produkt von Reihen: \ref{7.16}
\item Stetigkeit: \ref{P9}, \ref{9.1}
}

\subsection*{2014-11-17 (Mo)}
\items{
\item Vektorräume: Skript \textit{lineare Algebra} und \href{https://de.wikipedia.org/wiki/Vektorraum}{\textit{\underline{Wikipedia}}}
\items{
\item Vektor: \href{https://de.wikipedia.org/wiki/Vektor}{\textit{\underline{Wikipedia}}}
\item Skalar: \href{https://de.wikipedia.org/wiki/Skalar\_(Mathematik)}{\textit{\underline{Wikipedia}}}
}
\item Supremum, Infimum: \ref{3.10}
\item \href{https://de.wikipedia.org/wiki/Demokratismus}{\textit{Demokratismus}}, \href{https://de.wikipedia.org/wiki/Politische\_Korrektheit}{\textit{Political correctness}}, \href{https://de.wikipedia.org/wiki/Polygamy\_Porter}{\textit{Polygamy Porter}}, \href{https://de.wikipedia.org/wiki/Sildenafil}{\textit{Viagra}}, \href{https://de.wikipedia.org/wiki/Indignation}{\textit{\#Aufschrei}}
\item Stetige Funktionen: \ref{P9}
}

\subsection*{2014-11-20 (Do)}
\items{
\item \textit{Blondinenwitze, political correctness, "`man darf ja nix mehr sagen!"'}
\item Vektorräume: Skript \textit{lineare Algebra} und \href{https://de.wikipedia.org/wiki/Vektorraum}{\textit{\underline{Wikipedia}}}
\items{
\item Vektor: \href{https://de.wikipedia.org/wiki/Vektor}{\textit{\underline{Wikipedia}}} 
\item Skalar: \href{https://de.wikipedia.org/wiki/Skalar\_(Mathematik)}{\textit{\underline{Wikipedia}}}
}
\item Cauchy-Folge \& Vollständigkeit: $(x_n)_{n \in \N}$ Folge in Vektorraum $V$ heißt CF,\\
$\forall \eps > 0\ N_\eps \in \N\ \forall n,m > N_\eps ||x_n-x_m||<\eps$;\\
$V$ heißt vollständig, falls jede CF konvergiert.
\item Satz von Weierstraß: \ref{5.12}, \href{https://de.wikipedia.org/wiki/Satz\_von\_Bolzano-Weierstra\%C3\%9F}{\textit{\underline{Wikipedia}}}
\item \textit{"'Früher war alles besser, früher hatten Matheprofs noch Tafelwischer!"'}
\item $\eps-\delta$-Kriterium: \ref{9.1}
}

\subsection*{2014-11-24 (Mo)}
\items{
\item \textit{Sesamstraße, Muppet-Show, "`Der Mann ein Frosch, die Frau ein Schwein"', political correctness}
\item Stetigkeit: \ref{P9}
\items{
\item $\eps-\delta$-Kriterium: \ref{9.1}
}
\item Vollständigkeitssatz von Weierstraß: $(\eps([a,b]),||\cdot||_\infty)$ ist vollständig.\\
\textit{Im Forster nachgucken oder googlen - ich hab nichts unter dem Namen gefunden.}
\item Gleichmäßige Stetigkeit: \href{https://de.wikipedia.org/wiki/Gleichm\%C3\%A4\%C3\%9Fige\_Stetigkeit}{\textit{\underline{Wikipedia}}}
\item Zwischenwertsatz: \ref{9.18}
}

\subsection*{2014-11-27 (Do)}
\items{
\item Abgeschlossenheit, Kompaktheit: \ref{Intervalle}, \ref{AbgeschlossenKompakt}
\item Satz von Maximum und Minimum: \ref{9.20}
\item Punktweise Konvergenz: \ref{13.1}
\item Gleichmäßige Konvergenz: \ref{13.2}
\item Monotonie und Injektivität: \ref{10.2}
\item \textit{Systemfrage, \href{https://de.wikipedia.org/wiki/Bologna-Prozess}{Bologna}, Kritik an übertriebenen Regelwerken, Prüfungen doch lieber am Ende der Semesterferien, stattdessen "`hin zum Schlechteren"'}
\item Umkehrfunktion, Bijektivität: \ref{4.6}, \ref{4.7}, \ref{10.3}
\item Exponentialfunktion: \ref{8.8}, \ref{8.9}
\item Logarithmus (naturalis): \ref{10.5}
\item Allgemeine Potenzrechnung: \ref{Potenzfunktionen}
}

\subsection*{2014-12-01 (Mo)}
\items{
\item \textit{Drogen, "`was soll man studieren"', "`Politiker koksen"', alternative Medizin, "`Deutschland hat keine Verfassung"' (Grundgesetz $\neq$ Verfassung), "`Würde"' (Art. 1 (1) GG) nicht definiert, ebensowenig "`Geld"'}
\item Komplexe Zahlen: \ref{P6}
\items{
\item Geometrische Darstellung komplexer Zahlen: \ref{GeoKompZE}
\item Rechnen mit komplexen Zahlen: \ref{6.1}
\item Körper der komplexen Zahlen ($\C$): \ref{6.2}
\item Realteil, Imaginärteil: \ref{6.3}
\item Konjugiert komplexe Zahl, Betrag: \ref{6.4}
}
\item Sinus, Cosinus: \ref{SinCos}
\items{
\item Eigenschaften: \ref{8.17}
\item Grafische Darstellung: \ref{SinCosGraph}
}
}

\subsection*{2014-12-04 (Do)}
\items{
\item \textit{Kiffen, \href{https://de.wikipedia.org/wiki/Dimethyltryptamin}{DMT}, Nikotin: höchste Abhängigkeit, Heroin: Harte Entzugsphase, LSD: geringere Abhängigkeit, Gras: nur bei Überdosis abhängig, "`Ich finde, man sollte alle Drogen legalisieren. Wenn Sie 18 sind, sollten Sie selbst entscheiden dürfen."', Wirtschaftliche Vorteile (Steuern), hohe Qualität der "`Ware"' bei Legalisierung}
\item Komplexe Zahlen: \ref{P6}
\item Integration: \ref{P14}
\item Umkehrfunktion: \ref{4.6}
\items{
\item Ableitung: \ref{11.8}
}
\item Relationen: \href{https://de.wikipedia.org/wiki/Relation\_(Mathematik)}{\textit{\underline{Wikipedia}}}
\item Äquivalenzrelationen: \href{https://de.wikipedia.org/wiki/\%C3\%84quivalenzrelation}{\textit{\underline{Wikipedia}}}
}\newpage

\subsection*{2014-12-08 (Mo)}
\items{
\item \textit{"'Stell dir vor, es ist Krieg, und keiner geht hin"', "`Deutschland wird am Hindukusch verteidigt"' für'n Arsch, Spiegel Online lesen macht keinen kritischen Bildungsbürger}
\item Differentialrechnung: \ref{P11}
\items{
\item Ableitung: \ref{11.1}
\item Differenzierbarkeit $\Ra$ Stetigkeit: \ref{11.3}
\item Summe, Produkt, Differenzierbarkeit: \ref{11.4}
\item Kettenregel: \ref{11.6}
\item Ableitung der Umkehrfunktion: \ref{11.8}
\items{
\item \underline{Beweis (Klausuraufgabe):}\nl
$f$ stetig bei $x_0$, $f(x_0) > 0 \Ra\ \exists \delta > 0:\ f(x) > 0\ \forall (x-x_0) < \delta$.\nl
$\frac{\frac{1}{f(x)}-\frac{1}{f(x_0)}}{x-x_0} = \frac{\frac{f(x_0)-f(x)}{f(x)f(x_0)}}{x-x_0} = \underbrace{\frac{-1}{f(x)f(x_0)}}_{\Ra -\frac{1}{f'(x_0)}} \underbrace{\frac{f(x)-f(x_0)}{x-x_0}}_{\Ra f'(x_0)}$\nl
Quotientenregel: $\frac{NAZ-ZAN}{N^2}$ (\textit{"`Nenner Ableitung Zähler minus Zähler Ableitung Nenner durch Nenner Quadrat"'})\nl
$\left(\frac{f}{g}\right)'(x_0)=\frac{g(x_0)f'(x_0)-f(x_0)g'(x_0)}{(g'(x_0))^2}$\nl
Schreibweise: $\frac{df}{dx} = f'$
}
}
\item \textit{"'Auschwitz-Keule"', "`Nazi-Keule"', Kissinger, Fischer und co. "`geistige Elite"', "`Ich habe damals den Schröder in Hannover getroffen, auf der Straße. Wissen Sie, was ich zu ihm gesagt habe? 'Sie Arschloch!'. Diese Zivilcourage müssen Sie haben!"'}
\item Potenzreihen: \ref{13.8}
\items{
\item Konvergenzradius: \ref{13.10}
\item Ableitung von Potenzreihen: \ref{13.14}
}
}\newpage

\subsection*{2014-12-11 (Do)}
\items{
\item \textit{Studentische Mitbestimmung, Basisdemokratie, Dezentralität statt Bologna, "`Wollen Sie lieber ein flexibles Hauptstudium, wo Sie lernen, was Sie wollen, oder wollen Sie nur Zertifikate erlangen?"', Überbürokratisierung, "`Wie viel wollen wir uns vorschreiben lassen?"'}
\item \textit{"`Früher waren überall in der Uni noch Wasserspender"', lieber kostenfreie Trinkwasserbrunnen statt "`Nestle und CocaCola"'}
\item \textit{Bindestrichstudiengänge sind Mist - lieber einen vernünftigen Studiengang zuende studieren}
\item \textit{lieber richtiges Studium Generale, "`wo Sie studieren, was Sie interessiert, nicht, was die Professoren interessiert!"'}
\item \textit{Science Slam Mumpitz, "`Rechtsesoterischer, grobintellektueller Größenwahnsinniger"', "`Antisemit"', Nazi-Keule}
\item \textit{\textbf{"`Beteiligen Sie sich! Machen Sie Ihren Mund auf! Gehen Sie zu den Fachschaften!"'}}
\item Differentialrechnung: \ref{P11}
\items{
\item Produktregel, Quotientenregel: \ref{11.4}
\item Ableitung von Polynom: \ref{11.5}
\item Kettenregel: \ref{11.6}
}
\item \textit{\href{https://de.wikipedia.org/wiki/Alfred\_Herrhausen}{Alfred Herrhausen}, \href{https://de.wikipedia.org/wiki/Detlev\_Rohwedder}{Detlev Karsten Rohwedder} - beide von der \href{https://de.wikipedia.org/wiki/Rote\_Armee\_Fraktion}{RAF} ermordet}
\item \textit{"'Der Deutsche Depp muss alles bezahlen in Europa"' $\Ra$ Nationalismus, "`'Den Krötz kann man nicht verstehen, der spricht ja Dialekt!' - Du verstehst mich nicht, weil du'n Idiot bist!"'}
\item \textit{Pädagogische Erkenntnisse der letzten 20 Jahre sind Mist}
\item Differentialrechnung: \ref{P11}
\items{
\item Lokale Extrema: \ref{11.10}, \ref{11.15}
\item Ableitungen von Umkehrfunktionen: \ref{11.8}
\item Umkehrfunktion von $\exp$ ($\ln$ / $\log$): Beispiel von \ref{11.8}
\item Umkehrfunktionen von $\sin$ ($\arcsin$) und von $\cos$ ($\arccos$): \ref{12.7}
}
}\newpage

\subsection*{2014-12-15 (Mo)}
\items{
\item \textit{Deutschlandreise, Dialekte des Landes, in Dresden ist die Exponentialfunktion "`ö hoch x"', Bielefeld: "`Freunde der komplexen Zahlen kommen in Bielefeld wegen des hohen Imaginärteils der Stadt voll auf ihre Kosten"'}
\item \textit{"`Monsters of Poetry"' (\href{http://www.amazon.de/Monsters-Poetry-Die-Anthologie-Tour/dp/3941552023}{\underline{Amazon-Link}}), "`Mit gutem Wein kann man nichts falsch machen"'}
\item \textit{\textbf{Es gibt keine Übungsblätter über die Weihnachtsferien: "`Sie sollen in den zwei Wochen entspannen, nicht arbeiten!"'}}
\item \textit{\href{http://de.ria.ru/}{Ria Novosti}: "`Noch nie Satire $\Ra$ rechts?"', Gérard Depardieu trinkt 14 Flaschen Rotwein am Tag und knallt Löwen ab, Helmut Kohl hält BND für nutzlos}
\item \textit{"'Wie schmeckt eigentlich Löwenfleisch?"'}
\item \textit{In Franken (Prof. Dr. kommt aus Franken) sind die Kneipen und -gänger um 1 dicht}
\item \textit{Asterix \& Obelix, Analogien zum Unileben ("`Sebigbos"', "`Dekan"' $\Lra$ "`Derkannix"')}
\item \textit{Christian Wulff, "`Deutschland ist ein christlich geprägtes Land"' Mumpitz}
\item \textit{"'Die maßgebenden Menschen: Sokrates, Buddha, Konfuzius, Jesus"' (\href{http://www.amazon.de/Die-ma\%C3\%9Fgebenden-Menschen-Sokrates-Konfuzius/dp/3492201261}{\underline{Amazon-Link}})}
\item \textit{Kohl rechnet ab: Wulff "`Verräter"', "`Null"', Merkel "`lungert herum"', Geißler "`hinterfotzig"', Scheel "`charakterliche Null"' und "`nichts als NSDAP-Mitglied"', allgemein viel Kritik an Bundespräsidenten (\href{http://www.amazon.de/Verm\%C3\%A4chtnis-Die-Kohl-Protokolle-Heribert-Schwan/dp/3453200772}{\underline{Amazon-Link zur Kohl-Biographie}})}
\item \textit{CDU war mal "`Bildungs- und Professorenpartei"', "`hat mal liberales Gedankengut vertreten"'}
\item \textit{Atlantikbrücke, "`Alternativlosigkeit"', Deutschland hat keine Verfassung, "`Ich kann das Verantwortungsgeschwafel nicht mehr hören!"', sollte endlich mal neutral und pazifistisch werden - stattdessen mischen wir uns in Kriege ein und liefern Waffen in Krisengebiete, Einmarschieren "`darf nicht sein"'}
\item \textit{Hussein, Gaddafi - "`alles Hitlers"', Putin "`der nächste Hitler?"'}
\item Potenzreihen: \ref{13.8}
\items{
\item Konvergenzradius: \ref{13.10}, Berechnung: \ref{13.12}
\item Wurzelkriterium: \href{https://de.wikipedia.org/wiki/Wurzelkriterium}{\textit{\underline{Wikipedia}}}
}
\item Mittelwertsatz der Differentialrechnung, Satz von Rolle: \ref{11.12}
}\newpage

\subsection*{2014-12-18 (Do)}
\items{
\item \textit{Lektüre:}
\items{
\item \textit{Schiller, "`Sämtliche Gedichte"', Insel Verlag (\href{http://www.amazon.de/S\%C3\%A4mtliche-Gedichte-Balladen-Friedrich-Schiller/dp/3458172408/ref=sr\_1\_1?ie=UTF8\&qid=1418898008\&sr=8-1\&keywords=Schiller+Gedichte}{\underline{Amazon-Link}})}
\items{
\item \textit{"`Der Handschuh"', \href{http://www.literaturwelt.com/werke/schiller/handschuh.html}{\underline{Link}}}
\item \textit{"`Die Worte des Wahns"', \href{http://www.deutschelyrik.de/index.php/die-worte-des-wahns.html}{\underline{Link}}}
\item \textbf{Anmerkung des Mitschreibers:} \textit{"`Schiller"', Wise Guys, \href{https://www.youtube.com/watch?v=HptAN-B523w}{\underline{YouTube-Link}}}
}
\item \textit{Friedrich Nietzsche, "`Gedichte"', Diogenes Verlag, 16,80 DM (\href{http://www.amazon.de/Gedichte-Friedrich-Nietzsche/dp/3257227418/ref=sr\_1\_1?ie=UTF8\&qid=1418898076\&sr=8-1\&keywords=nietzsche+gedichte+diogenes}{\underline{Amazon-Link}})}
\item \textit{Albert Camus, "`Der Mythos von Sisyphos"', ro ro ro Verlag, 10,90 DM (\href{http://www.amazon.de/Mythos-Sisyphos-Albert-Camus/dp/3499227657/ref=sr\_1\_1?ie=UTF8\&qid=1418898121\&sr=8-1\&keywords=camus+mythos}{\underline{Amazon-Link}})}
\item \textit{Giolo Mann, "`Deutsche Geschichte des 19. \& 20. Jahrhunderts"', Fischer Verlag (\href{http://www.amazon.de/Deutsche-Geschichte-19-20-Jahrhunderts/dp/359611330X/ref=sr\_1\_1?ie=UTF8\&qid=1418898172\&sr=8-1\&keywords=mann+deutsche+geschichte}{\underline{Amazon-Link}})}
\items{
\item \textit{Thomas Mann, "`Der Zauberberg"' (\href{http://www.amazon.de/Zauberberg-Roman-Thomas-Mann/dp/3596294339/ref=sr\_1\_1?ie=UTF8\&qid=1418901461\&sr=8-1\&keywords=zauberberg}{\underline{Amazon-Link}}, \href{http://de.wikipedia.org/wiki/Der\_Zauberberg}{\underline{Wikipedia-Link}})}
\item \textit{Thomas Mann, "`Doktor Faustus"' (\href{http://www.amazon.de/Doktor-Faustus-deutschen-Tonsetzers-Leverk\%C3\%BChn/dp/3596294282/ref=sr\_1\_1?ie=UTF8\&qid=1418901563\&sr=8-1\&keywords=doktor+faustus}{\underline{Amazon-Link}}, \href{https://de.wikipedia.org/wiki/Doktor\_Faustus}{\underline{Wikipedia-Link}})}
}
}
\item \textit{Interessante und lustige Leute:}
\items{
\item \textit{\href{http://de.wikipedia.org/wiki/Marcel\_Reich-Ranicki}{\underline{Marcel Reich-Ranicki}}, Buchkritiker}
\item \textit{\href{http://de.wikipedia.org/wiki/Stan\_Laurel}{\underline{Stan Laurel}}, Komiker ("`Dick und Doof"')}
\item \textit{\href{http://de.wikipedia.org/wiki/Sigi\_Zimmerschied}{\underline{Sigi Zimmerschmied}}, Kabarettist}
\item \textit{\href{http://de.wikipedia.org/wiki/Gerhard\_Polt}{\underline{Gerhard Polt}}, Kabarettist}
\item \textit{\href{http://de.wikipedia.org/wiki/Karl\_Valentin}{\underline{Karl Valentin}}, Komiker}
\item \textit{\href{http://de.wikipedia.org/wiki/J\%C3\%BCrgen\_von\_Manger}{\underline{Jürgen von Manger}}, Schauspieler, Kabarettist und Komiker, Bühnenfigur "`Adolf Tegtmeier"', CD: Wunderbar (\href{http://www.amazon.de/Wunderbar-Manger-J\%C3\%BCrgen-Von/dp/B00008VAHI/ref=sr\_1\_1?ie=UTF8\&qid=1418898710\&sr=8-1\&keywords=wunderbar+manger}{\underline{Amazon-Link}})}
\item \textit{\href{https://de.wikipedia.org/wiki/Wilhelm\_Tell}{\underline{Wilhelm Tell}}, "`Asterix bei den Schweizern"' (\href{http://www.amazon.de/Asterix-16-bei-den-Schweizern/dp/3770436164/ref=sr\_1\_1?ie=UTF8\&qid=1418898888\&sr=8-1\&keywords=asterix+bei+den+schweizern}{\underline{Amazon-Link}})}
}
\item \textit{Prof. Dr. hat mit 18 seine Entschuldigungen selbst unterschrieben und Deutsch geschwänzt, "`Wenn die anderen in der Schule waren, saß Krötz alleine im Café"'}
\item \textit{Prof. Dr. hält \href{https://de.wikipedia.org/wiki/Urknall}{\underline{Urknalltheorie}} für Mumpitz, "`Nur weil ein Mann im Rollstuhl sitzt, soll er plötzlich schlau sein?!"'}
\item \textit{Mit offenen Augen durch die Welt gehen}
\item \textit{Prof. Dr. kritisiert vereinheitlichtes Europa "`der Vorschriften aus Brüssel"', lieber ein Europa nationaler Identitäten, "`Kontroversen aushalten - das ist Europa. Nicht Einheitsbrei"', "`Europa braucht keinen Euro"'}
\item \textit{Mit Mathe sind wir fast durch, nur noch \href{http://de.wikipedia.org/wiki/Riemannsches_Integral}{\underline{Riemann-Integrale}} nach den Ferien}
\item \textit{\textbf{Frohe Feiertage! Entspannt euch! :)}}
}

\subsection*{2015-01-05 (Mo)}
\items{
\item Krümmung: \href{https://de.wikipedia.org/wiki/Kr\%C3\%BCmmung}{\underline{\textit{Wikipedia}}}
\item Satz (Kriterium für lokale Extrema): \ref{11.15}
\item Konvexität: \ref{11.17}
\item Konvexitätskriterium: \ref{11.18}
\item Ungleichungen
\items{
\item AGM (arithm.-geom. Mittel): \href{https://de.wikipedia.org/wiki/Arithmetisch-geometrisches\_Mittel}{\underline{\textit{Wikipedia}}}
\item CSU (Cauchy-Schwarzsche Ungleichung): \href{https://de.wikipedia.org/wiki/Cauchy-Schwarzsche\_Ungleichung}{\underline{\textit{Wikipedia}}}
\item DEU (Dreiecksungleichung): \ref{3.8}, \ref{6.7}, \href{https://de.wikipedia.org/wiki/Dreiecksungleichung}{\underline{\textit{Wikipedia}}}
\item Verallgemeinerung: Hölder-Ungleichung: \href{https://de.wikipedia.org/wiki/H\%C3\%B6lder-Ungleichung}{\underline{\textit{Wikipedia}}}
}
}

\subsection*{2015-01-08 (Do)}
\items{
\item Normen auf Vektorräumen: \href{https://de.wikipedia.org/wiki/Norm\_(Mathematik)\#Vektornormen}{\underline{\textit{Wikipedia}}}
\items{
\item Maximumsnorm: \href{https://de.wikipedia.org/wiki/Maximumsnorm\#Als\_Vektornorm}{\underline{\textit{Wikipedia}}}
}
}

\subsection*{2015-01-12 (Mo)}
\items{
\item Integration: \ref{P14}
\item Riemannsches Integral, Riemann-Summe: \href{https://de.wikipedia.org/wiki/Riemannsches\_Integral}{\textit{\underline{Wikipedia}}}
\item Treppenfunktionen: \ref{14.1}
\item Eigenschaften des Integrals von Trepenfunktionen: \ref{14.2}
}

\subsection*{2015-01-15 (Do)}
\items{
\item \textit{Ausgefallen}
}

\subsection*{2015-01-19 (Mo)}
\items{
\item Riemannsches Integral, Riemann-Summe: \href{https://de.wikipedia.org/wiki/Riemannsches\_Integral}{\textit{\underline{Wikipedia}}}
\item Eigenschaften des Integrals: \ref{14.2}
\items{
\item Zusätzlich: $f: [a,c] \to \R$ integrierbar $\Lra \forall a < b < c:\ f|_[a,b], f|_[b,c]$ integrierbar und $\int_a^c f\ dx = \int_a^b f\ dx + \int_b^c f\ dx$
}
\item $f$ stetig $\Ra$ integrierbar
\item $f$ monoton $\Ra$ integrierbar
\item Mittelwertsatz der Integralrechnung: \ref{14.8}
\item Stammfunktion: \ref{14.9}
\item Hauptsatz der Differential- und Integralrechnung: \ref{14.11} $\Ra$ Integrieren = "`Aufleiten"'
\item Substitutionsregel: \ref{14.14}
\item Partielle Integration: \ref{14.13}
}

\subsection*{2015-01-22 (Do)}
\items{
\item Hauptsatz der Differential- und Integralrechnung: \ref{14.11}
\item Taylor: \ref{P15}
\item Restglied-Abschätzung (Lagrange-Formel): \ref{15.3}
\item Approximation n-ter Ordnung
}

\subsection*{2015-01-26 (Mo)}
\items{
\item Uneigentliche Integrale: \ref{UneigentlicheIntegrale}
\item Gamma-Funktion: \ref{14.19}
\items{
\item Eigenschaften: \ref{EigenschaftenGamma}
}
\item Logarithmische Konvexität: \href{https://de.wikipedia.org/wiki/Logarithmische\_Konvexit\%C3\%A4t}{\textit{\underline{Wikipedia}}}
}

\subsection*{2015-01-29 (Do)}
\items{
\item Satz von Bohr-Mollerup: \href{https://de.wikipedia.org/wiki/Gammafunktion\#Der\_Satz\_von\_Bohr-Mollerup}{\textit{\underline{Wikipedia}}}
\item Gauß'sche Limesdarstellung: \href{https://de.wikipedia.org/wiki/Gammafunktion\#Weitere\_Darstellungsformen}{\textit{\underline{Wikipedia}}}
\item Wallis'sches Produkt: \href{https://de.wikipedia.org/wiki/Wallissches\_Produkt}{\textit{\underline{Wikipedia}}}
\item Stirling-Formel: \href{https://de.wikipedia.org/wiki/Stirlingformel}{\textit{\underline{Wikipedia}}}
}

\subsection*{2015-02-02 (Mo)}
\items{
\item Trapezregel: \href{https://de.wikipedia.org/wiki/Trapezregel}{\textit{\underline{Wikipedia}}}
\item Wallis'sches Produkt: \href{https://de.wikipedia.org/wiki/Wallissches\_Produkt}{\textit{\underline{Wikipedia}}}
}