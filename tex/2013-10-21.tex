% Kopfzeile beim Kapitelanfang:
\fancypagestyle{plain}{
%Kopfzeile links bzw. innen
\fancyhead[L]{\Large Vorlesung 3 (21.10.2013)}
%Kopfzeile rechts bzw. außen
\fancyhead[R]{}}
%Kopfzeile links bzw. innen
\fancyhead[L]{\Large Vorlesung 3 (21.10.2013)}
%Kopfzeile rechts bzw. außen
\fancyhead[R]{}
% **************************************************

\subsection*{Folgerung}
$\bigbin{n}{k} \in \N_0$ für $1 \le k \le n$

\subsection*{Beweis}
Ziehe $k$ Kugeln aus einer Urne mit $n$ nummerierten Kugeln ohne Zurücklegen (zunächst unter Beachtung der Reihenfolge):\nl
1. Zug: $n$ Möglichkeiten\\
2. Zug: $n-1$ Möglichkeiten\\
$k$. Zug: $n-k+1$ Möglichkeiten\nl
Insgesamt: $n \cdot (n-1) \cdot \hdots \cdot (n-k+1)$ Möglichkeiten\nl
Nach \emph{Satz 2.6} kommt dabei jede $k$-elementige Teilmenge in $k!$ verschiedenen Anordnungen vor (Reihenfolge der Kugeln).\nl
$\Rightarrow$ Die Anzahl der $k$-elementigen Teilmengen ist: $\frac{n \cdot {n-1} \cdot \hdots \cdot (n-k+1)}{k!} = \bigbin{n}{k}$ \qed

\subsection*{Wichtige Anwendung}\label{binFormel}
Seien $x,y \in \R$ und $n \in \N_0$.\\
Was ist dann $(x+y)^n$?\nl
$(x+y)^0=1$ (per Definition)\\
$(x+y)^1=x+y$\\
$(x+y)^2=x^2+2xy+y^2$ (Binomische Formel)\\
$(x+y)^3=x^3+3x^2y+3xy^2+y^3$\nl
Vermutung: Die Koeffizienten sind gerade Binomialkoeffizienten.

\newpage

\section{Binomischer Satz}\label{2.10}
Seien $x,y \in \R$ und $n \in \N_0$.\\
Dann ist \fbox{$(x+y)^n = \bigsum_{k=0}^n \bigbin{n}{k} x^k y^{n-k}$}\\
Insbesondere gilt für $y=1$: $(x+1)^n = \bigsum_{k=0}^n \bigbin{n}{k} x^k$

\subsection*{Beweis mit vollständiger Induktion}
\en{
\item Induktionsanfang: $n=0$\\
$(x+y)^0 = \bigbin{0}{0} x^0 y^0 = 1$ \ok
\item Induktionsvoraussetzung: Die Formel gilt für ein beliebiges, festes $n$.
\item Induktionsschluss: $n \rightarrow n+1$\\
$(x+y)^{n+1}=(x+y)\cdot(x+y)^n=(x+y)\cdot\left(\bigsum_{k=0}^n \bigbin{n}{k} x^k y^{n-k}\right)$\\
$=x \cdot \left(\bigsum_{k=0}^n \bigbin{n}{k} x^k y^{n-k}\right)+y \cdot \left(\bigsum_{k=0}^n \bigbin{n}{k} x^k y^{n-k}\right)$\\
$=\bigsum_{k=0}^n \bigbin{n}{k} x^{k+1} y^{n-k} + \bigsum_{k=0}^n \bigbin{n}{k} x^k y^{n-k+1}$\\
$=\bigsum_{k=1}^{n+1} \bigbin{n}{k-1} x^k y^{n-(k-1)} + \bigsum_{k=0}^n \bigbin{n}{k} x^k y^{n-k+1}$\\
$=\underbrace{y^{n+1}}_{k=0} + \underbrace{x^{n+1}}_{k=n+1} + \bigsum_{k=1}^n \underbrace{\left(\bigbin{n}{k} + \bigbin{n}{k-1}\right)}_{\bigbin{n+1}{k}} x^k y^{n+1-k}$\\
$=\bigsum_{k=0}^{n+1} \bigbin{n+1}{k} x^k y^{n+1-k}$ \ok
}
\qed

\newpage

\chapter{Die reellen Zahlen}\label{P3}
Ein Ziel bei der Erweiterung von Zahlenbereichen ist die Lösbarkeit von Gleichungen.\\
$\N \rightarrow \Z$: Löse $x+n=m$ mit $n,m \in \N_0$\\
$\Z \rightarrow \Q$: Löse $x \cdot n=m$ mit $n,m \in \Z$ und $n \neq 0$\\
Aber: $x^n=m$ mit $n,m \in \N$ hat in der Regel keine Lösung in $\Q$!

\section*{Beispiel (aus der Zentralübung)}
Es gibt kein $x \in \Q$ für $x^2=2$.\\
Abhilfe: Erweiterung $\Q \rightarrow \R$ (reelle Zahlen).\\
Im $\R$ hat $x^2=2$ zwei Lösungen: $x=\pm \sqrt{2}$

\section*{Axiomatische Einführung des reellen Zahlenraums}\label{AxiomeReelleZahlen}
\emph{(geht zurück auf \href{https://de.wikipedia.org/wiki/David_Hilbert}{David Hilbert})}\nl
Wir geben eine Reihe von grundlegenden Eigenschaften für $\R$ an:
\en{
\item Körperaxiome
\item Anordnungsaxiome
\item Vollständigkeitsaxiom (sichert u.a. die Lösbarkeit von $x^n=m$ mit $n,m \in \N$ ab)
}
Das Körper- sowie das Anordnungsaxiom werden übrigens auch von $\Q$ erfüllt.\nl
Man kann zeigen (wichtiger Satz), dass es genau eine Menge $\R$ mit diesen Eigenschaften (bis auf Umbenennungen) gibt.\\
Es gibt präzise Konstruktionen $\N \rightarrow \Z \rightarrow \Q \rightarrow \R$ so dass $\N \subset \Z \subset \Q \subset \R$ ist.

\section{Vorbemerkung: Quantoren}\label{3.1}
Gegeben seien Aussagen $P(x)$ mit $x \in X$ ($X$ sei eine Menge),\\
z.B. $X=\N$ und $P(x)=x \text{ ist gerade}$.\nl
\begin{tabular}{c|c}
Aussage & Schreibweise \\ 
\hline 
Für alle $x \in X$ gilt $P(x)$ & $\forall x \in X : P(x)$\\
Es gibt mindestens ein $x \in X$ mit $P(x)$ & $\exists x \in X : P(x)$\\
Es gibt genau ein $x \in X$ mit $P(x)$ & $\exists! x \in X : P(x)$\\
Es gibt kein $x \in X$ mit $P(x)$ & $\nexists x \in X : P(x)$
\end{tabular}\nl
Aus dem Vorwort wissen wir: $\forall$ ist der Allquantor, $\exists$ der Existenzquantor.

\subsection*{Beispiele}
$\exists n \in \N : n \ge 2$, $\nexists x \in \Q : x^2 = 2$, $\forall x \in \Q \exists n \in \N : x \cdot n \in \Z$

\newpage

\phantomsection
\addcontentsline{toc}{section}{Definition: Körperaxiome}
\section*{Definition: Körperaxiome}\label{Koerperaxiome}
Auf der Menge $\R$ sind zwei Rechenoperationen $+$ (die Addition) und $\cdot$ (die Multiplikation) erklärt, so dass $(\R, +, \cdot)$ ein Körper ist.

\section{Definition: Körper}\label{3.2}
Ein Körper ist eine Menge $K$ mit zwei Operationen $+$ und $\cdot$, so dass gilt:
\begin{enumerate}[label=(K\arabic*)]
\item Assoziativgesetz:\\
$\forall x,y,z \in K : (x+y)+z = x+(y+z)$ (Assoziativität für die Addition)\\
$\forall x,y,z \in K : (x \cdot y) \cdot z = x \cdot (y \cdot z)$ (Assoziativität für die Multiplikation)
\item Kommutativgesetz:\\
$\forall x,y \in K : x+y = y+x$ (Kommutativität für die Addition)\\
$\forall x,y \in K : x \cdot y = y \cdot x$ (Kommutativität für die Multiplikation)
\item Existenz neutraler Elemente:\\
$\forall x \in K \exists 0 \in K : x+0 = x$ (Nullelement)\\
$\forall x \in K \exists 1 \in K : x \cdot 1 = x$ (Einselement)
\item Existenz von Inversen:\\
$\forall x \in K \exists y \in K : x+y = 0$ (additives Inverses von $x$)\\
$\forall x \in K \setminus \{0\} \exists z \in K : x \cdot z = 1$ (multiplikatives Inverses von $x$)
\item Distributivgesetz:\\
$\forall x,y,z \in K : x \cdot (y+z) = x \cdot y + x \cdot z$
\end{enumerate}

\subsection*{Anmerkung}
Jedem Paar $(x,y) \in K \times K$ wird genau ein Element $x+y \in K$ bzw. $x \cdot y = xy \in K$ zugeordnet.

\section{Folgerungen}\label{3.3}
Sei $K$ ein Körper.
\en{
\item $0$ und $1$ sind eindeutig bestimmt.\\
Beweis für $0$ (für $1$ analog): Sei auch $0'$ neutral bezüglich der Addition.\\
$\Rightarrow 0=0+0' \underset{K2}{=} 0'+0 \underset{K3}{=} 0'$
\item $y$ und $z$ in (K4) sind (bei festem $x$) eindeutig bestimmt.\\
Beweis für $y$ (Addition): Sei $x+y'=0=x+y$\\
$\Rightarrow y \underset{K3}{=} y+0 = y+(x+y') \underset{K1}{=} (y+x)+y' \underset{K2}{=} (x+y)+y' = 0+y' \underset{K2}{=} y'$\\
Bezeichnung: $-x$ ist das additive Inverse von $x$ und $x^{-1} = \frac{1}{x}$ das multiplikative Inverse von $x \neq 0$
\item $a,b \in K \Rightarrow$ die Gleichung $a+x=b$ hat eine eindeutige Lösung $x=b+(-a) =: b-a$\\
Falls $a \neq 0 \Rightarrow$ auch $a \cdot x = b$ hat eine eindeutige Lösung $x=a^{-1} \cdot b =: \frac{b}{a}$\\
Beweis für die Addition: $a+x=b \Leftrightarrow a+x+(-a)=b+(-a) \Leftrightarrow (a+(-a))+x=b-a \Leftrightarrow x=b-a$\\
Analog für die Multiplikation.
\item $-(-x)=x$ und falls $x \neq 0$ gilt: $(x^{-1})^{-1}=x$\\
Beweis für das additive Inverse: $(-x)+x \underset{K2}{=} x+(-x)=0 \underset{K2}{\Rightarrow} -(-x)$
\item $-(x+y)=-x-y$ und falls $x,y \neq 0$ gilt: $(xy)^{-1} = x^{-1} y^{-1}$ (Beweis in Übung)
\item $\forall x \in K : x \cdot 0 = 0$\\
Denn: $\frac{x \cdot 0}{a}+x \cdot 0 = x \cdot (0+0) = \frac{x \cdot 0}{b} \Rightarrow x \cdot 0 = x \cdot 0 - x \cdot 0 = 0$
\item $xy = 0 \Leftrightarrow x=0 \vee y=0$
\item $(-x)y=-xy$\\
Insbesondere: $(-1)y=-y$
}
Beweise von 7. und 8. in der Übung.