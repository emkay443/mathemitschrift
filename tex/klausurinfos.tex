% Kopfzeile beim Kapitelanfang:
\fancypagestyle{plain}{
%Kopfzeile links bzw. innen
\fancyhead[L]{\Large Informationen zur Klausur}
%Kopfzeile rechts bzw. außen
\fancyhead[R]{}}
%Kopfzeile links bzw. innen
\fancyhead[L]{\Large Informationen zur Klausur}
%Kopfzeile rechts bzw. außen
\fancyhead[R]{}
% **************************************************

\phantomsection
\addcontentsline{toc}{chapter}{Infos zur 1. Klausur WS 2013/14}
\chapter*{Infos zur 1. Klausur WS 2013/14}
\section*{Grundsätzliches zur Klausur}
\items{
%\item 20. Februar 2014, 13:30-15:30 Uhr in der Sporthalle SP2 (\href{http://www.uni-paderborn.de/fileadmin/uni-homepage/images/lageplan-03.13.jpg?dur=357}{\emph{\underline{Lageplan}}})
%\item Es wird wahrscheinlich \textbf{6 Aufgaben} geben
\item Keine Aufgabe ist optional, alle sind zu bearbeiten und gehen vollständig in die Zielnote mit ein
\item Ein beidseitig handbeschriebener DIN-A4 Spicker ist als Hilfsmittel zugelassen
\item Weitere Hilfsmittel (u.a. Taschenrechner) sind nicht zugelassen!
}

\section*{Aufgaben der ersten Klausur}
\emph{Folgendes ist aus meinem Gedächtnisprotokoll der ersten Klausur.\\
Dieses, andere sowie Klausuren der Vorjahre können in der Fachschaft (E1.311) eingesehen werden.}\nl

\noindent \textbf{Aufgabe 1: Rekursive Folgen}\newline
Gegeben: eine rekursive Definition für $(a_n)$
\begin{enumerate}[label=(\alph*)]
\item Beweis mit vollständiger Induktion, dass sich $a_n$ zwischen zwei Werten aufhält
\item Zeigen Sie Monotonie der Folge
\item Prüfen Sie, ob die Folge konvergiert, und wenn, dann berechnen Sie den Grenzwert
\end{enumerate}

\bigskip \hrule \bigskip

\noindent \textbf{Aufgabe 2: Komplexe Zahlen}
\begin{enumerate}[label=(\alph*)]
\item Wandeln Sie zwei gegebene, komplexe Zahlen um in das Format $x+yi$
\item Berechnen Sie von einer gegebenen, komplexen Zahl $z$ (nicht im Format $x+yi$) den Betrag $|z|$
\end{enumerate}

\bigskip \hrule \bigskip

\noindent \textbf{Aufgabe 3: Differentialrechnung}\newline
Gegeben: eine Funktion $f$
\begin{enumerate}[label=(\alph*)]
\item Skizzieren Sie den Graphen der Funktion
\item Wo ist die Funktion differenzierbar? Berechnen Sie dort die Ableitung
\end{enumerate}

\bigskip \hrule \bigskip

\noindent \textbf{Aufgabe 4: Konvergenzradius}\newline
Gegeben: Potenzreihe
\begin{enumerate}[label=(\alph*)]
\item Berechnen Sie den Konvergenzradius dieser Reihe
\item Wo konvergiert diese Reihe?
\end{enumerate}

\newpage

\noindent \textbf{Aufgabe 5: Kurvendiskussion, Trigonometrie}\newline
Zeigen Sie, dass eine gegeben Funktion $f$ genau eine Nullstelle in einem gegebenen, halboffenen Intervall hat

\bigskip \hrule \bigskip

\noindent \textbf{Aufgabe 6: Kurvendiskussion, Logarithmus}
\begin{enumerate}[label=(\alph*)]
\item Berechnen Sie für $x \to \infty$ den Grenzwert von einem gegebenen Term mit Logarithmus Naturalis
\item Untersuchen Sie die Funktion diesen Terms auf lokale und globale Extrema
\end{enumerate}

\bigskip \hrule \bigskip

\noindent \textbf{Aufgabe 7: Integralrechnung}
\begin{enumerate}[label=(\alph*)]
\item Berechnen Sie den Wert eines gegebenen Integrals
\item Zeigen/widerlegen Sie, dass ein gegebenes Integral konvergiert
\end{enumerate}

%\items{
%\item Folgen, Grenzwerte, rekursive Folgen (wie in Zwischenklausur)
%\item Komplexe Zahlen (z.B. Einheitswurzel)
%\item Potenzreihen, Konvergenzradien
%\item Reihen, Reihenkonvergenz, Leibniz-Kriterium
%\item Stetigkeit
%\item Ableitungen
%\item Hauptsatz der Integral- und Differentialrechnung
%\item Zwischen-, Mittelwertsatz
%\item Integralrechnung, uneigentliche Integrale, partielle Integrale, Substitution
%\item Taylorpolynome
%}