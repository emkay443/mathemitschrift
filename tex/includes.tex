\usepackage[ngerman]{babel}
\usepackage[utf8]{inputenc}

% Klickbare Verknüpfungen
\usepackage[hidelinks]{hyperref}

% Hübsche Header
\usepackage{fancyhdr}
\pagestyle{fancy}

% Zusätzliche Symbole (u.a. Widerspruchsblitz)
\usepackage{marvosym}

% Grafiken importieren
\usepackage{graphicx}

% Farben
\usepackage{xcolor}

% Zeichenpaket (Graphen, Zahlenstrahl, etc.)
\usepackage{tikz}
\usetikzlibrary{decorations.pathreplacing}
\usetikzlibrary{lindenmayersystems}
\usetikzlibrary{patterns}
\newcommand{\streiche}[1]{%
    \tikz[baseline=(tocancel.base)]{
        \node[inner sep=0pt,outer sep=0pt] (tocancel) {#1};
        \draw[red] (tocancel.south west) -- (tocancel.north east);
    }%
}%

\newlength{\hatchspread}
\newlength{\hatchthickness}
\newlength{\hatchshift}
\newcommand{\hatchcolor}{}
% declaring the keys in tikz
\tikzset{hatchspread/.code={\setlength{\hatchspread}{#1}},
         hatchthickness/.code={\setlength{\hatchthickness}{#1}},
         hatchshift/.code={\setlength{\hatchshift}{#1}},% must be >= 0
         hatchcolor/.code={\renewcommand{\hatchcolor}{#1}}}
% setting the default values
\tikzset{hatchspread=3pt,
         hatchthickness=0.4pt,
         hatchshift=0pt,% must be >= 0
         hatchcolor=black}
% declaring the pattern
\pgfdeclarepatternformonly[\hatchspread,\hatchthickness,\hatchshift,\hatchcolor]% variables
   {custom north west lines}% name
   {\pgfqpoint{\dimexpr-2\hatchthickness}{\dimexpr-2\hatchthickness}}% lower left corner
   {\pgfqpoint{\dimexpr\hatchspread+2\hatchthickness}{\dimexpr\hatchspread+2\hatchthickness}}% upper right corner
   {\pgfqpoint{\dimexpr\hatchspread}{\dimexpr\hatchspread}}% tile size
   {% shape description
    \pgfsetlinewidth{\hatchthickness}
    \pgfpathmoveto{\pgfqpoint{0pt}{\dimexpr\hatchspread+\hatchshift}}
    \pgfpathlineto{\pgfqpoint{\dimexpr\hatchspread+0.15pt+\hatchshift}{-0.15pt}}
    \ifdim \hatchshift > 0pt
      \pgfpathmoveto{\pgfqpoint{0pt}{\hatchshift}}
      \pgfpathlineto{\pgfqpoint{\dimexpr0.15pt+\hatchshift}{-0.15pt}}
    \fi
    \pgfsetstrokecolor{\hatchcolor}
%    \pgfsetdash{{1pt}{1pt}}{0pt}% dashing cannot work correctly in all situation this way
    \pgfusepath{stroke}
   }


% Linie unten
\renewcommand{\footrulewidth}{0.5pt}
% Kopfzeile Linie oben
\renewcommand{\headrulewidth}{0.5pt}
% Fußzeile
\fancyfoot[C]{- \thepage\ -}
% Kopfzeilen fürs Inhaltsverzeichnis:
% Kopfzeile links bzw. innen
\fancyhead[L]{\Large Inhaltsverzeichnis}
% Kopfzeile rechts bzw. außen
\fancyhead[R]{\Large{}}
% Kopfzeile beim Kapitelanfang:
\fancypagestyle{plain}{
	% Kopfzeilen fürs Inhaltsverzeichnis:
	% Kopfzeile links bzw. innen
	\fancyhead[L]{\Large Inhaltsverzeichnis}
	% Kopfzeile rechts bzw. außen
	\fancyhead[R]{\Large{}}
}

\usepackage{tocloft}
\renewcommand\cftsecnumwidth{1.2cm}
\renewcommand\cftchapnumwidth{0.75cm}

\setcounter{chapter}{-1} %Überschriftennummer neu setzen (zahl der bereits vergebenen Nummern)
\renewcommand{\thechapter}{§\arabic{chapter}}
\renewcommand{\thesection}{\arabic{chapter}.\arabic{section}}
\setcounter{tocdepth}{2}


% Mathegedöns
\usepackage{amsmath}%Mathebiblothek
\usepackage{amssymb} %Sonderzeichen (z.B. \checkmark)
\usepackage{enumitem}%Aufzählunden formatieren
\usepackage{polynom}

% Kurzbefehle

%Zahlenmengen start
\newcommand{\C}{\mathbb{C}}%komplexe Zahlen
\newcommand{\R}{\mathbb{R}}%reelle Zahlen
\newcommand{\Q}{\mathbb{Q}}%rationale Zahlen
\newcommand{\Z}{\mathbb{Z}}%ganze Zahlen
\newcommand{\N}{\mathbb{N}}%Natürliche Zahlen
\newcommand{\Lsg}{\mathbb{L}}%Lösungsmenge 
\newcommand{\F}{\mathbb{F}}%Körper (z.B. F_2)
\newcommand{\K}{\mathbb{K}}%Körper (z.B. K_2)
%Zahlenmengen end

\newcommand{\A}{\mathcal{A}} % Geschwungenes A
\newcommand{\E}{\mathcal{E}} % Geschwungenes E/großes Epsilon
\newcommand{\Pow}{\mathcal{P}} % Potenzmenge (engl. Power Set)
\newcommand{\eps}{\varepsilon}
\newcommand{\ok}{\hfill{\checkmark}}
\newcommand{\qed}{\hfill{q.e.d.}}
\newcommand{\wspruch}{\Lightning}
\newcommand{\nl}{\\[8pt]} %für Absätze
\newcommand{\es}{\emptyset}
\newcommand{\rem}[1]{} %Blockweise kommentieren ;)
%Aufzählung Zahlen
\newcommand{\en}[1]{\begin{enumerate}[label=(\arabic*)] #1 \end{enumerate}}
%Aufzählung Zahlen (ohne Klammer, mit Punkt)
\newcommand{\enk}[1]{\begin{enumerate}[label=\arabic*.] #1 \end{enumerate}}
%Aufzählung Buchstaben
\newcommand{\ena}[1]{\begin{enumerate}[label=(\alph*)] #1 \end{enumerate}}
%Aufzählung Römische Buchstaben
\newcommand{\enr}[1]{\begin{enumerate}[label=(\roman*)] #1 \end{enumerate}}
%Aufzählung Römische Großbuchstaben
\newcommand{\enR}[1]{\begin{enumerate}[label=(\Roman*)] #1 \end{enumerate}}
%Aufzählung
\newcommand{\items}[1]{\begin{itemize} #1 \end{itemize}}

% Große Summe, Produkt, Binomialkoeffizient, ...
\newcommand{\bigsum}{\displaystyle\sum}
\newcommand{\bigprod}{\displaystyle\prod}
\newcommand{\bigbin}{\displaystyle\binom}
\newcommand{\biglim}{\displaystyle\lim}
\newcommand{\bigfrac}{\displaystyle\frac}

% Große Klammern
\newcommand{\bigbrackets}[1]{\left( #1 \right)}
\newcommand{\bigbraces}[1]{\left\{ #1 \right\}}
% Auskommentieren, um große Schreibweise zu forcieren
\everymath{\displaystyle}

% "Nach Induktionsvoraussetzung" Gleich
\newcommand{\iv}{\underset{I.V.}{=}}

% Pfeile
\newcommand{\Ra}{\Rightarrow}
\newcommand{\La}{\Leftarrow}
\newcommand{\Lra}{\Leftrightarrow}
\newcommand{\rsa}{\rightsquigarrow}
\newcommand{\mto}{\mapsto}

% Grad von ... (Polynome)
\newcommand{\grad}{\text{grad }}

% Realteil, Imaginärteil
\newcommand{\RE}{\text{Re }}
\newcommand{\IM}{\text{Im }}