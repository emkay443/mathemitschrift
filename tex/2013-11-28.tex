% Kopfzeile beim Kapitelanfang:
\fancypagestyle{plain}{
%Kopfzeile links bzw. innen
\fancyhead[L]{\Large Vorlesung 14 (28.11.2013)}
%Kopfzeile rechts bzw. außen
\fancyhead[R]{}}
%Kopfzeile links bzw. innen
\fancyhead[L]{\Large Vorlesung 14 (28.11.2013)}
%Kopfzeile rechts bzw. außen
\fancyhead[R]{}
% **************************************************
\subsection*{Zusatz}
Die Dezimaldarstellung von $x$ ist eindeutig, wenn man ausschließt, dass $a_k=9$ für fast alle $k$ ist, also $x=\ldots,\ldots \overline{9}$\nl
Hier ohne Beweis.

\subsection*{Bemerkung}
Statt der Basis $10$ kann man eine beliebige Zahle $b \in \N$, $b \ge 2$ als Basis wählen.\\
Analog zu Satz \ref{8.2} gilt: Jedes $x \in \R$ hat eine so genannte \underline{$b$-adische Entwicklung}
$$x = \pm \sum_{k=-N}^\infty a_k b^{-k} \text{ mit } N \in \N_0, a_k \in \{0, \ldots, b-1\}$$
$b = 2$: Dualsystem, $a_k \in \{0,1\} \to$ Binärdarstellung von $x$\\
$b = 8$: Oktalsystem (vor 1980 in der Informatik)

\section{Definition: Abzählbare Mengen, Mächtigkeit}\label{8.3}
\en{
\item Zwei Mengen $X, Y$ heißen \underline{gleichmächtig}, kurz $X \sim Y$, $\Lra \exists$ bijektive Abbildung $f: X \to Y$
\item $X$ ist abzählbar unendlich $\Lra X \sim \N$
\item $X$ ist abzählbar $\Lra X$ ist endlich (auch $X=\emptyset$ möglich) oder abzählbar unendlich
\item $X$ ist überabzählbar $\Lra X$ ist nicht abzählbar
}
Abzählbar unendlich sind zum Beispiel:
\items{
\item $\N$
\item $\N_0, f: \N \to \N_0, n \to n-1$ ist bijektiv
\item $\Z,$ bijektives $f: \N \to \Z: \begin{array}{l l l l l l l} 1 & 2 & 3 & 4 & 5 & 6 & \ldots \\ 0 & 1 & -1 & 2 & -2 & 3 & \ldots \end{array}$
}
Achtung: $\N \subsetneq \N_0 \subsetneq \Z$, aber alle drei Mengen sind gleichmächtig!

\newpage

\section{Lemma}\label{8.4}
$X \neq \emptyset$ ist abzählbar $\Lra \exists$ surjektives $f: \N \to X$, das heißt, mit $x_i = f(i)$ gilt: $X = \{x_1, x_2, x_3, \ldots\}$

\subsection*{Beweis}
\items{
\item[``$\Ra$''] o.E. $X=\{x_1, \ldots, x_n\}$ endlich\\
Definiere $f: \N \to X: \begin{array}{l l} f(i) := x_1 & \text{für } 1 \le i \le n \\ f(i) := x_n & \text{für } i > n \end{array}$, $f$ ist surjektiv
\item[``$\La$''] Sei $f: \N \to X$ surjektiv, o.E. $X$ unendlich\\
$X = \{f(1), f(2), f(3), \ldots\}$, $f(i)=f(j)$ für $i \neq j$ ist möglich\\
Konstruiere daraus neue Abzählung $X=\{x_1, x_2, x_3, \ldots\}$, bei der bereits gezählte Elemente übersprungen werden, das heißt $\underbrace{f(1)}_{=x_1}=\streiche{f(2)}=\streiche{f(3)}, \underbrace{f(4)}_{=x_2}=\streiche{f(5)} \ldots$\\
$x_1 := f(1), x_2 := f(4), 4 :=$ kleinster Index mit $f(4) \neq x_1$\\
$x_2 := f(i_2), i_2 :=$ kleinster Index mit $f(i_2) \notin \{x_1, x_2\}$ usw...\\
Liefert Bijektion $\N \to X, i \mto x_i$\\
Konsequenz: $X$ ist abzählbar, $Y \subseteq X \Ra Y$ ist abzählbar (anschaulich klar).
}

\section{Satz}\label{8.5}
Seien $X_n$ ($n \in \N$) abzählbare Mengen $\Ra X = \bigcup_{n \in \N} X_n$ ist abzählbar.

\subsection*{Beweis}
o.E. $X_n \neq \emptyset \forall n \in \N$\\
Sei $X_1 = \{x_{11}, x_{12}, x_{13}, \ldots \}$\\
$X_2 = \{x_{21}, x_{22}, x_{23}, \ldots \}$\\
etc...\nl
$\begin{array}{l l l l l l l}
\overset{1}{x_{11}} & \rightarrow & \overset{2}{x_{12}} & \ & \overset{6}{x_{13}} & \rightarrow & \overset{7}{x_{14}} \\
\ & \swarrow & \ & \nearrow & \ & \ & \ \\
\overset{3}{x_{21}} & \ & \overset{5}{x_{22}} & \ & \ldots & \ & \ldots \\
\downarrow & \nearrow & \ & \ & \ & \ & \ \\
\overset{4}{x_{31}} & \ & \ldots & \ & \ldots & \ & \ldots
\end{array}$\nl
Abzählung von $X$: 

\section{Korollar}\label{8.6}
$\Q$ ist abzählbar.

\subsection*{Beweis}
$\Q = \bigcup_{n \in \N} \underbrace{\left\{\frac{k}{n}: k \in \Z\right\}}_{\text{abzählbar}}$ 

\newpage

\section{Satz: Überabzählbarkeit der reellen Zahlen}\label{8.7}
$\R$ ist überabzählbar.

\phantomsection
\addcontentsline{toc}{subsection}{Beweis (Cantorsches Diagonalverfahren)}
\subsection*{Beweis (Cantorsches Diagonalverfahren)}
Wir zeigen, dass bereits $(0,1)$ überabzählbar ist.\\
Angenommen, $(0,1)$ sei abzählbar: $(0,1) = \{x_1, x_2, x_3, \ldots\}$\\
$x_n$ in Dezimaldarstellung (ohne $\overline{9}$)\\
$x_1 = 0, a_{11} a_{12} a_{13} \ldots$\\
$x_2 = 0, a_{21} a_{22} a_{23} \ldots$\\
$x_3 = 0, a_{31} a_{32} a_{33} \ldots$\\
$\vdots$\nl
Definiere Dezimalzahl $z \in (0,1)$ wie folgt:\\
$x_1 = 0, {\color{red} \mathbf{a_{11}}} a_{12} a_{13} \ldots$\\
$x_2 = 0, a_{21} {\color{red} \mathbf{a_{22}}} a_{23} \ldots$\\
$x_3 = 0, a_{31} a_{32} {\color{red} \mathbf{a_{33}}} \ldots$\\
$\vdots$\\
$z := 0,z_1 z_2 z_3$ mit $z_n := \left\{ \begin{array}{l l} 5 & \text{falls } a_{nn} \neq 5 \\ 4 & \text{falls } a_{nn} = 5 \end{array} \right.$\\
$\Ra z \in (0,1): z_n \neq a_{nn} \Ra \underset{\text{Eindeutigkeit der Darstellung}}{\Ra} z \neq x_n \forall n$

\subsection*{Beispiel}
Jedes Programm ist eine endliche Folge von Symbolen aus einer endlichen Menge $A$ (Alphabet).\\
Nach Satz \ref{8.5} $\Ra \{$ Programme (über $A$)$\}$ ist abzählbar\\
$X = \{f: \N \to \{0,1\}\}$ Menge der $0,1$-Folgen\\
$f \in X$ ist berechenbar $\Lra \exists$ Programm, das zu einer Eingabe $n$ die Ausgabe $f(n)$ liefert.\\
$\Ra \{f \in X: f \text{ berechenbar}\}$ ist abzählbar\nl
Aber: $X$ ist überabzählbar! (Übung)\\
Also: $\exists f: \N \to \{0,1\}$, die nicht berechenbar sind!

\section{Satz: Die Exponentialfunktion}\label{8.8}
Für jedes $z \in \C$ ist die \underline{Exponentialfunktion}\\
\fbox{$exp(z) := \sum_{n=0}^\infty \frac{z^n}{n!}$} absolut konvergent.

\subsection*{Beweis}
$z = 0 \Ra $Konvergenz klar; $exp(0)=1$\\
$z \neq 0$: Quotientenkriterium: $a_n = \frac{z^n}{n!} \neq 0$\\
$\left|\frac{a_{n+1}}{a_n}\right| = \left|\frac{z^{n+1}}{(n+1)!} \cdot \frac{n!}{z^n}\right| = \frac{1}{n+1} \cdot |z| \to 0$ für $n \to \infty$\nl
Nach Korollar \ref{7.14} folgt die Behauptung.

\newpage

\section{Definition: Exponentialfunktion}\label{8.9}
Die Funktion $exp: \C \to \C, z \mto exp(z)$ heißt \underline{Exponentialfunktion}.

\subsection*{Beachte}
$x \in \R \Ra exp(x)=\sum_{n=0}^\infty \frac{x^n}{n!} \in \R$

\subsection*{Spezielle Werte}
\items{
\item $exp(0)=1$
\item \fbox{$exp(1)=\sum_{n=0}^\infty \frac{1}{n!} =: e \approx 2,718$} \underline{Eulersche Zahl}
}

\section{Satz: Funktionalgleichung}\label{8.10}
$\forall z,w \in \C:$ \fbox{$exp(z+w) = exp(z) \cdot exp(w)$}

\subsection*{Beweis}
Exponentialreihe ist absolut konvergent $\Ra$ Cauchy-Produkt bildbar:\\
$exp(z) \cdot exp(w) = \left(\sum_{j=0}^\infty \frac{z^j}{j!}\right) \cdot \left(\sum_{k=0}^\infty \frac{w^k}{k!}\right) = \sum_{n=0}^\infty c_n$\\
mit $c_n = \sum_{j=0}^n \frac{z^j}{j!} \cdot \frac{w^{n-j}}{(n-j)!} = \frac{1}{n!} \cdot \sum_{j=0}^n \binom{n}{j} z^j w^{n-j} = \frac{1}{n!} (z+w)^n$ (Binomische Formel)\\
$\Ra exp(z) \cdot exp(w) = \sum_{n=0}^\infty \frac{1}{n!} \cdot (z+w)^n = exp(z+w)$ \qed

\newpage

\section{Folgerungen}\label{8.11}
\en{
\item $exp(z) \neq 0 \forall z \in \C$ und $exp(-z) = \frac{1}{exp(z)}$\\
Denn: $1=exp(0) = exp(z+(-z)) \underset{\text{\ref{8.10}}}{=} exp(z) \cdot exp(-z)$ \ok
\item \fbox{$x \in \R \Ra exp(x) > 0$}\\
Denn: Für $x \ge 0$: $exp(x) = 1 + \sum_{n=1}^\infty \underbrace{\frac{x^n}{n!}}_{\ge0} \ge 1$\\
Für $x < 0$: $exp(x) \underset{(1)}{=} \frac{1}{exp(-x)} > 0$ \ok
\item $z \in \C, n \in \Z \Ra exp(nz) = (exp(z))^n$\\
Insbes. (mit $z=1$): \fbox{$exp(n) = e^n$} $\forall n \in \Z$\\
Denn: Für $n \in \N_0$: Induktion nach $n$\\
$n=0$: $(exp(z))^0 = 1 = exp(0 \cdot z)$\\
$n \to n+1$: $exp((n+1)z) = exp(nz+z) \underset{\text{\ref{8.10}}}{=} exp(nz) \cdot exp(z) \iv (exp(z))^{n+1}$ \ok\\
Für $n < 0$: $exp(nz) \underset{\text{(1)}}{=} \frac{1}{exp(-nz)} \underset{-n > 0}{=} \frac{1}{(exp(z))^{-n}} = (exp(z))^n$ \qed
\item $p \in \Z, q \in \N \Ra exp\left(\frac{p}{q}\right) = \sqrt[q]{e^p} =: e^{p/q}$\\
Denn: $\left(exp\left(\frac{p}{q}\right)\right)^q \underset{\text{(3)}}{=} exp\left(q \cdot \frac{p}{q}\right) = exp(p) = e^p \underset{exp\left(\frac{p}{q}\right) >0}{\Ra} exp\left(\frac{p}{q}\right) = \sqrt[q]{e^p}$ \qed
}

\subsection*{Schreibweise (motiviert durch (4))}
\fbox{$e^z := exp(z)$} für $z \in \C$\\
Damit: $e^0 = 1$, $e^{z+w} = e^z \cdot e^w$ (FG \ref{8.10}), $e^{-z} = \frac{1}{e^z}$