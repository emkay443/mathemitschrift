% Kopfzeile beim Kapitelanfang:
\fancypagestyle{plain}{
%Kopfzeile links bzw. innen
\fancyhead[L]{\Large Vorlesung 6 (31.10.2013)}
%Kopfzeile rechts bzw. außen
\fancyhead[R]{}}
%Kopfzeile links bzw. innen
\fancyhead[L]{\Large Vorlesung 6 (31.10.2013)}
%Kopfzeile rechts bzw. außen
\fancyhead[R]{}
% **************************************************
\subsection*{Beispiele}\label{BeispieleAbbildungen}
\en{
\item $X=\{$Studenten an der Uni Paderborn$\}, f: X \to \N, x \mapsto $ Alter von $x$
\item $f: \R \to \R, x \mapsto x^2$\\
$\Gamma_f:$ Normalparabel\\
$f(\R)=\{x \in \R, x>0\}$\\
Wenn einem $x \in X$ mehrere $y \in Y$ zugeordnet werden können, ist es keine Funktion!
\item \underline{Lineare Funktionen}:\\
$f: \R \to \R, x \mapsto ax+b$ ($a,b \in \R$)\\
$\Gamma_f$: Gerade, $a$: Steigung, $b$: Achsenabschnitt
\item \underline{Polynomfunktionen}:\\
Funktion der Form $p: \R \to \R, p(x)=a_n x^n + \hdots + a_1 x + a_0$\\
mit $n \in \N_0$ und $a_1 , \hdots , a_n \in \R$ Koeffizienten
\item \underline{Floorfunktion (Gauß-Klammer)}:\\
$\lfloor . \rfloor = \R \to \R$\\
$\lfloor x \rfloor :=$ größtes $k \in \Z$ mit $k \le x$
\item \underline{Identische Abbildung}:\\
$id_x: X \to X, x \mapsto x$
\item $P :=$ Menge aller Sortierprogramme für endliche Listen\\
$L: P \times \N \to \R$\\
$L(p,n) :=$ max. Laufzeit, die Programm $p$ zum Sortieren einer Liste der Länge $n$ braucht
}
Für die Charakterisierung einer Abbildung ist neben der Abbildungsvorschrift auf der Definitionsbereich $X$ wichtig.

\subsection*{Beispiel}
$f: \R \to \R, f(x)=|x|$ und $g: [-1,1] \to \R, g(x)=|x|$ sind verschiedene Abbildungen, denn $g$ ist eine Restriktion von $f$.

\section{Definition: Bild, Urbild}\label{4.2}
\begin{enumerate}[label=(\roman*)]
\item Sei $f: X \to Y$ eine Abbildung. Sei $A \subseteq X$.\\
$f(A) := \{f(x) : x \in A\}$ ist das \underline{Bild von $A$ unter $f$}.\\
$f(X) \subseteq Y$ ist der Wertebereich von $f$.
\item Sei $B \subseteq Y$.\\
$f^{-1}(B) := \{x \in X : f(x) \in B\}$ ist das \underline{Urbild von $B$ unter $f$}.\\
$f^{-1}$ ist hierbei \underline{keine} eigenständige Funktion, sondern nur eine Bezeichnung!
\end{enumerate}

\subsection*{Beispiel}
$f: \Z \to \Z: x \mapsto x^2$\\
$f(\{-2,5\})=\{4,25\}$\\
$f^{-1}(\{4,25\})=\{\pm 2,\pm 5\}$\\
$f^{-1}(\{3\})=\es$\\
$f^{-1}(\{4\})=\{\pm 2\}$\\
$f^{-1}(\{3,4\})=\{\pm 2\}$

\newpage

\section{Definition: Komposition von Abbildungen}\label{4.3}
Seien $f: X \to Y$ und $g: Y \to Z$ Abbildungen.\\
Die Komposition (Verknüpfung, Verkettung) von $f$ und $g$ ist die Abbildung:\\
$g \circ f: X \to Z, x \mapsto g(f(x))$ ($g$ nach $f$)

\subsection*{Beispiel}
$f,g : \R \to \R, f(x)=2x+1, g(x)=x^2$\\
$\left. \begin{array}{l}
(g \circ f)(x) = (2x+1)^2=4x^2+4x+1\\
(f \circ g)(x) = 2x^2+1
\end{array} \right \} \Rightarrow f \circ g \neq g \circ f$

\section{Satz: Assoziativität der Komposition}\label{4.4}
Seien $f: X \to Y$, $g: Y \to Z$ und $h: Z \to W$ Abbildungen.\\
$\Rightarrow h \circ (g \circ f) = (h \circ g) \circ f$

\subsection*{Beweis}
$(h \circ (g \circ f))(x)=h((g \circ f)(x))=h(g(f(x)))=((h \circ g) \circ f)(x)$ \qed

\section{Definition: Eigenschaften von Abbildungen (Injektivität, Surjektivität, Bijektivität)}\label{4.5}
$f: X \to Y$ heißt:
\en{
\item \underline{injektiv}, falls es zu jedem $y \in Y$ \underline{höchstens} ein $x \in X$ gibt mit $f(x)=y$
\item \underline{surjektiv}, falls es zu jedem $y \in Y$ \underline{mindestens} ein $x \in X$ gibt mit $f(x)=y$
\item \underline{bijektiv}, falls $f$ sowohl injektiv als auch surjektiv ist, also falls es zu jedem $y \in Y$ \underline{genau} ein $x \in X$ gibt mit $f(x)=y$
}

\subsection*{Also}
$f$ injektiv $\Leftrightarrow \forall x_1, x_2 \in X$ gilt: wenn $f(x_1)=f(x_2)$ dann $x_1=x_2$\\
$f$ surjektiv $\Leftrightarrow \forall y \in Y \exists x \in X: y=f(x) \Leftrightarrow f(X)=Y$\\
$f$ bijektiv $\Leftrightarrow \forall y \in Y \exists! x \in X: y=f(x)$

\subsection*{Anmerkungen zur Hausübung}
$2^n-1$ prim $\Rightarrow n$ prim\\
Beweis durch Widerspruch:\\
Angenommen, $n=pq$ und $p,q \neq 1$\\
$2^n-1=2^{pq}-1=(\underbrace{2^p}_{x})^q -1=x^q -1$

\newpage

\subsection*{Beispiele}
$f: \N \to \N$ drei Varianten:
\begin{enumerate}[label=(\roman*)]
\item $f(n)=n+1$ ist injektiv, aber nicht surjektiv
\item $f(n)=\left\{ \begin{array}{l l}
1 & \text{falls } n=1 \\
n-1 & \text{falls } n \ge 2
\end{array} \right.$ ist surjektiv, aber nicht injektiv
\item $f(n)=\left\{ \begin{array}{l l}
n-1 & \text{falls } n \text{ gerade} \\
n+1 & \text{falls } n \text{ ungerade}
\end{array} \right.$ ist bijektiv
\end{enumerate}
Sei $f:X \to Y$ bijektiv, das heißt $\forall y \in Y \exists! x \in X : f(x)=y$.\\
Wir können dann $g: Y \to X$ definieren durch: $g(y) := x$ falls $y=f(x)$.\\
Damit ist $g(f(x))=x \forall x \in X$ und $f(g(y)) = y \forall y \in Y$.\\
Das heißt: $g \circ f = id_x$ und $f \circ g = id_y$.

\section{Definition: Umkehrabbildung}\label{4.6}
$g$ heißt \underline{Umkehrabbildung von $f$}.\\
Bezeichnung: $g=f^{-1}$ \emph{(hier ist $f^{-1}$ tatsächlich eine Funktion!)}\nl
Damit (falls $f$ bijektiv): $y=f(x) \Leftrightarrow x=f^{-1}(y)$

\section{Satz: Äquivalenz Bijektivität, Umkehrabbildung}\label{4.7}
Für $f: X \to Y$ sind äquivalent:
\begin{enumerate}[label=(\roman*)]
\item $f$ ist bijektiv
\item $\exists g: Y \to X: g \circ f = id_x, f \circ g = id_y$\\
In diesem Fall ist $g=f^{-1}$.
\end{enumerate}

\subsection*{Beweis}
\items{
\item[(i) $\Rightarrow$ (ii)] siehe oben
\item[(ii) $\Rightarrow$ (i)] $f$ ist injektiv, denn: Sei $f(x-1)=f(x_2) \Rightarrow x_1=g(f(x_1))=g(f(x_2))$\\
$f$ ist surjektiv, denn: Sei $y \in Y \Rightarrow y=f(g(y))$\\
$g$ ist in (ii) eindeutig, da $g(f(x))=x$ und $f$ surjektiv $\Rightarrow g=f^{-1}$ 
} \qed

\subsection*{Frage}
Sei $f: X \to Y$ bijektiv und $X,Y \subseteq \R$.\\
Was ist der Graph von $f^{-1}$?\\
$\Gamma_f = \{(x,f(x)): x \in X\}$\\
$\Gamma_{f^{-1}} = \{(f(x),x): x \in X\}$ entsteht durch Spiegelung an der Hauptdiagonalen $y=x$

\subsection*{Beispiele}
\en{
\item $f: \R \to \R, f(x)=x^2$ (Normalparabel)\\
$f$ ist weder surjektiv (da $x^2>0 \forall x \in \R$ also $f(\R) \subseteq [0,\infty)$) noch injektiv (da $f(-x)=f(x)$).\nl
$f(\R)=[0,\infty)$ denn $y \ge 0 \Rightarrow \exists! x \ge 0: x^2-y$\nl
Betrachte $g: [0,\infty) \to [0,\infty), g(x)=x^2$\\
$\Rightarrow g$ ist bijektiv mit Umkehrfunktion $g^{-1}(y)=\sqrt{y}$
\item $f: \R \to \R, f(x)=3x+2$ ist bijektiv mit $f^{-1}(y)=\frac{y-2}{3}$
}
