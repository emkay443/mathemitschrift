% Kopfzeile beim Kapitelanfang:
\fancypagestyle{plain}{
%Kopfzeile links bzw. innen
\fancyhead[L]{\Large Vorlesung 1 (14.10.2013)}
%Kopfzeile rechts bzw. außen
\fancyhead[R]{}}
%Kopfzeile links bzw. innen
\fancyhead[L]{\Large Vorlesung 1 (14.10.2013)}
%Kopfzeile rechts bzw. außen
\fancyhead[R]{}
% **************************************************
\chapter{Logik und Mengen}\label{P1}

\phantomsection
\addcontentsline{toc}{section}{Aussagen}
\section*{Aussagen}\label{Aussagen}
Eine math. \underline{Aussage} ist ein sprachlicher Ausdruck, dem genau ein Wahrheitswert $wahr (w)$ oder $falsch (w)$ zugeordnet werden kann. 

\subsection*{Beispiel}
\items{
\item $2\cdot3=6$ ist wahr
\item $1+2=4$ ist falsch
\item ``Tom ist ein guter Koch'' ist keine Aussage, da nicht eindeutig wahr oder falsch
}

\section{Verknüpfung von Aussagen}\label{1.1}
Die Verknüpfung von Aussagen liefert neue Aussagen.\\
Seien $P$ und $Q$ zwei Aussagen.
\en{
\item \underline{Konjunktion}: $P \wedge Q$ (sprich: $P$ und $Q$)\\
Ist wahr genau dann, wenn $P$ und $Q$ wahr sind.
\item \underline{Disjunktion}: $P \vee Q$ (sprich: $P$ oder $Q$)\\
Ist wahr genau dann, wenn $P$ oder $Q$ wahr ist (oder beide).
\item \underline{Negation}: $\neg P$ (sprich: nicht $P$)\\
Ist wahr genau dann, wenn $P$ falsch ist.
\item \underline{Implikation}: $P \Rightarrow Q$ (sprich: $P$ impliziert $Q$, aus $P$ folgt $Q$)\\
Ist falsch genau dann, wenn $P$ wahr, aber $Q$ falsch ist.\\
Ist immer wahr, wenn $P$ falsch ist.
\item \underline{Äquivalenz}: $P \Leftrightarrow Q := (P \Rightarrow Q) \wedge (Q \Rightarrow P)$\\
Ist wahr genau dann, wenn $P$ und $Q$ denselben Wahrheitswert haben.\\
Beispiel: Für ganze Zahlen $n$ gilt: $n^2 \le 4 \Leftrightarrow -2 \le n \le 2$
}

\subsection*{Wahrheitstafel}
\begin{tabular}{c|c|c|c|c|c|c}
$P$ & $Q$ & $P \wedge Q$ & $P \vee Q$ & $\neg P$ & $P \Rightarrow Q$ & $P \Leftrightarrow Q$ \\ 
\hline 
w & w & w & w & f & w & w \\
w & f & f & w & f & f & f \\
f & w & f & w & w & w & f \\
f & f & f & f & w & w & w
\end{tabular} 

\newpage

\section{Bezeichnung: Semantische Äquivalenz}\label{1.2}
Zwei Aussagen $P$ und $Q$ mit selber Wahrheitstafel heißen \underline{semantisch äquivalent}: $P \equiv Q$

\subsection*{Beispiele}
\en{
\item $\neg (\neg P) \equiv P$ denn:\\
\begin{tabular}{c|c|c}
$P$ & $\neg P$ & $\neg (\neg P)$ \\ 
\hline 
w & f & w \\
f & w & f
\end{tabular} 
\item \underline{Regeln von de Morgan}:\\
$\neg (P \wedge Q) = \neg P \vee \neg Q$\\
$\neg (P \vee Q) = \neg P \wedge \neg Q$\\
Erkenntnis: $\neg$ bindet stärker als $\wedge$ und $\vee$.
\item \underline{Distributivgesetze}:\\
$P \wedge (Q \vee R) = (P \wedge Q) \vee (P \wedge R)$\\
$P \vee (Q \wedge R) = (P \vee Q) \wedge (P \vee R)$
}

\newpage

\phantomsection
\addcontentsline{toc}{section}{Mengen}
\section*{Mengen}\label{Mengen}
Definition nach \href{https://de.wikipedia.org/wiki/Georg_Cantor}{G. Cantor}: \emph{``Eine Menge ist eine Zusammenfassung bestimmter, wohlunterschiedener Objekte unseres Denkens zu einem Ganzen. Diese Objekte heißen Elemente der Menge.''}\nl
Diese Definition ist \emph{vage}!\\
Pragmatischer Standpunkt: Eine Menge ist gebildet, wenn feststeht, welche Elemente ihr zugehörig sind.

\subsection*{Schreibweisen}
\items{
\item $x \in A$ falls $x$ Element der Menge $A$ ist
\item $x \notin A$ falls $x$ kein Element der Menge $A$ ist
\item $A \subseteq B$ (sprich: $A$ ist Teilmenge von oder gleich $B$) falls jedes Element aus $A$ auch in $B$ vorkommt
\item $B \supseteq A$ (sprich: $B$ ist Obermenge von oder gleich $A$)
\item $A = B$ falls $A \subseteq B$ und $B \subseteq A$
\item $\emptyset$ ist die leere Menge
\item $A \subset B$ (auch: $A \subsetneq B$) heißt, $A$ ist eine Teilmenge von, aber nicht gleich $B$
}

\subsection*{Beschreibung von Mengen}
\en{
\item \underline{Durch Aufzählung der Elemente, z.B.}:\\
$A=\{rot,blau\}$\\
$B=\{1,\{1,2\},\{1,2,\{1,2\}\}\}$\\
$\N=\{1,2,3,\hdots\}$ Menge der natürlichen Zahlen\\
$\N_0=\{0,1,2,3,\hdots\}$ Menge der natürlichen Zahlen inklusive der Null\\
$\Z=\{0,\pm 1,\pm 2,\hdots\}$ Menge der ganzen Zahlen
\item \underline{Durch eine charakterisierende Eigenschaft, z.B.}:\\
$A=\{n \in \Z : n \text{ ist gerade}\}$\\
$B=\{n \in \Z : \text{es gibt } k \in \Z \text{ mit } n=2k\}$\\
$P=\{p \in \N : p \text{ ist Primzahl}\}$
}

\newpage

\section{Definition: Teiler}\label{1.3}
Seien $m,n \in \Z$.\\
$m$ heißt \underline{Teiler von $n$} (kurz: $m \mid n$), falls es ein $k \in \Z$ gibt mit $n=k \cdot m$.\\
Sonst ist $m$ kein Teiler von $n$, also $m \nmid n$.

\subsection*{Beispiele}
\items{
\item $-3 \mid 6$, da es ein $k \in \Z$ gibt mit $6=k \cdot -3$, nämlich $k=-2$
\item $5 \nmid 6$, da es kein $k \in \Z$ gibt mit $6=k \cdot 5$
\item $p \in \N$ heißt \underline{Primzahl}, falls $p \neq 1$ und $p$ keine Teiler in $\N$ hat außer $1$ und $p$
}

\subsection*{Weitere Mengen von Zahlen}\label{Rationale Zahlen}
$\Q = \left \{ \frac{m}{n} : m,n \in \Z : n \neq 0 \right \}$ Menge der rationalen Zahlen\\
Rechnen in $\Q$ ist aus der Schule bekannt.\\
Wichtig:$\frac{a}{b} = \frac{c}{d} \Leftrightarrow ad = bc$\\
$\R$ Menge der reellen Zahlen

\subsection*{Inklusionen}
$\N \subset \N_0 \subset \Q \subset \R$

\section{Mengenoperationen}\label{1.4}
Seien $A$ und $B$ zwei Mengen.
\items{
\item $A \cup B := \{x : x \in A \vee x \in B \}$ Vereinigung von $A$ und $B$
\item $A \cap B := \{x : x \in A \wedge x \in B \}$ Durchschnitt von $A$ und $B$
\item $A \setminus B := \{x : x \in A \wedge x \notin B\}$ Komplement von $A$ und $B$
}

\section{Regeln von de Morgan}\label{1.5}
\en{
\item $A \cap (B \cup C) = (A \cap B) \cup (A \cap C)$
\item $A \cup (B \cap C) = (A \cup B) \cap (A \cup C)$
}

\subsection*{Beweis der ersten Regel von de Morgan}
$x \in A \cap (B \cup C) \Leftrightarrow (x \in A) \wedge (x \in (B \cup C)) \Leftrightarrow (x \in A) \wedge ((x \in B) \vee (x \in C))$\\
$\Leftrightarrow ((x \in A) \wedge (x \in B)) \vee ((x \in A) \wedge (x \in C)) \Leftrightarrow (x \in A \cap B) \vee (x \in A \cap C) \Leftrightarrow x \in (A \cap B) \cup (A \cap C)$ \\
\qed

\newpage

\section{Kartesisches Produkt von Mengen}\label{1.6}
Seien $A$ und $B$ zwei Mengen.\\
$A \times B := \{(a,b) : a \in A , b \in B \}$ Menge aller geordneten Paare $(a,b)$\\
Im Gegensatz zu $\{a,b\}=\{b,a\}$ kommt es beim kartesischen Produkt auf die Reihenfolge an!\\
Insbesondere: $(a,b)=(a',b') \Leftrightarrow a=a' \wedge b = b'$

\subsection*{Beispiele}
$A=\{1,2,3\}$\\
$B=\{\ast,\circ\}$\\
$A \times B = \{(1,\ast),(1,\circ),(2,\ast),(2,\circ),(3,\ast),(3,\circ)\}$

\section{Bezeichnung: Potenzmengen, Mächtigkeit}\label{1.7}
Sei $A$ eine Menge.
\en{
\item $\Pow(A) := \{B : B \subseteq A \}$ \underline{Potenzmenge von $A$}\\
Beispiel: $A=\{0,1\} \Rightarrow \Pow(A)=\{\emptyset,\{0\},\{1\},\{0,1\}\}$
\item Ist $A$ endlich, so kennzeichnet $|A|$ die \underline{Mächtigkeit}, die \underline{Anzahl der Elemente} von $A$.\\
Falls $A$ nicht endlich ist, so ist $|A|=\infty$
}
Es gilt: $|A| = n \in \N_0 \Rightarrow |\Pow(A)|=2^n$ (mit Konvention $2^0=1$)\\
Beachte: $A=\emptyset \Rightarrow \Pow(A)=\{\emptyset\}$